% -------------------------------------------------------
% Daten für die Arbeit
% Wenn hier alles korrekt eingetragen wurde, wird das Titelblatt
% automatisch generiert. D.h. die Datei titelblatt.tex muss nicht mehr
% angepasst werden.

\newcommand{\hsmasprache}{en} % de oder en für Deutsch oder Englisch
                              % Für korrekt sortierte Literatureinträge, noch preambel.tex anpassen

% Titel der Arbeit auf Deutsch
\newcommand{\hsmatitelde}{Erweiterung von zwei dimensionalen Morphogenese-Urothelium-Simulations Modellen in drei Dimensionen}

% Titel der Arbeit auf Englisch
\newcommand{\hsmatitelen}{Extension of two dimensions morphogenesis simulation models of the urothelium into three dimensions within the moduro simulation environment}

% Weitere Informationen zur Arbeit
\newcommand{\hsmaort}{Mannheim}    % Ort
\newcommand{\hsmaautorvname}{Thorsten} % Vorname(n)
\newcommand{\hsmaautornname}{Müller} % Nachname(n)
\newcommand{\hsmadatum}{13.03.2018} % Datum der Abgabe
\newcommand{\hsmajahr}{2018} % Jahr der Abgabe
\newcommand{\hsmafirma}{} % Firma bei der die Arbeit durchgeführt wurde
\newcommand{\hsmabetreuer}{Prof. Dr. Markus Gumbel, University of Applied Sciences Mannheim} % Betreuer an der Hochschule
\newcommand{\hsmazweitkorrektor}{PD. Dr. med. Philipp Erben, Clinic of Urology, University of Heidelberg} % Betreuer im Unternehmen oder Zweitkorrektor
\newcommand{\hsmafakultaet}{I} % I für Informatik
\newcommand{\hsmastudiengang}{IB} % IB IMB UIB IM MTB

% Zustimmung zur Veröffentlichung
\setboolean{hsmapublizieren}{true}   % Einer Veröffentlichung wird zugestimmt
\setboolean{hsmasperrvermerk}{false} % Die Arbeit hat keinen Sperrvermerk

% -------------------------------------------------------
% Abstract

% Kurze (maximal halbseitige) Beschreibung, worum es in der Arbeit geht auf Deutsch
\newcommand{\hsmaabstractde}{}

% Kurze (maximal halbseitige) Beschreibung, worum es in der Arbeit geht auf Englisch
\newcommand{\hsmaabstracten}{The bladder is one of many organs in humans and animals. It is coated by the urothelium, which ensures that the bladder contains only urine. With bladder cancer the urothelium is not able to ensure its barrier function.  \newline
Bladder cancer is one of the most common cancer types in men. Cancer is able to spread into the surrounding organs and muscles. In the case of bladder cancer, the cancer arises in the urothelium and then spreads into the neighboring organs as it grows. If bladder cancer spreads into the surrounding organs and muscles the affected person has almost no chance of healing. Therefore, it is important to recognize bladder cancer as early as possible. To do so insights on how it arises are required. To receive these insights several simulations in two dimensions were done. With these simulations first insights of the rise of bladder cancer are revealed. \newline
Because the urothelium is a three-dimensional object, the simulation should be extended by a third dimension. With this extension it is hoped to receive new insights on how bladder cancer arises.
}

\newcommand{\acknowledgement}{Before this bachelor thesis starts I have to make several expressions of thanks.\newline
The first one goes out to my supervisor Markus Gumbel. I am grateful for the opportunity to write and complete this thesis as well as the patience, several ideas as I was stuck and of course for the criticism I received during this experience. It was a huge support during this journey as well as to come to a point with this bachelor thesis. 

I am thankful to Angelo Torelli, who still have enough knowledge in the project to support newcomers as I was one. With the answers regarding my questions it was possible to clarify uncertainties, which helped me to fully understand the different functionalities of the project as well as to save time by studying the project. 

The next reward goes out to my second supervisor Philipp Erben and Julian Debatin. With the willingness of my second supervisor this bachelor thesis was possible. In addition both provided me the possibility to ask them about the biology of the urothelium. 

Last but not least, I am obliged to Saskia Müller, Ute Müller and Stefan Rasemann. With the immense time effort they put in to this thesis, they helped me not only to improve my English but more important to take this paper to the next level.}