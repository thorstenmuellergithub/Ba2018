% Document type and used packages
\documentclass[open=right, % Kapitel darf nur auf rechten Seite beginnen
    paper=A4,               % DIN-A4-Papier
    a4paper,                % DIN-A4-Papier
    12pt,                   % Schriftgöße
    headings=small,         % Kleine Überschriften
    headsepline=true,       % Trennlinie am Kopf der Seite
    footsepline=false,      % Keine Trennlinie am Fuß der Seite
    bibliography=totoc,     % Literaturverzeichnis in das Inhaltsverzeichnis aufnehmen
    twoside=on,             % Doppelseitiger Druck - auf off stellen für einseitig
    DIV=7,                  % Verhältnis der Ränder zum bedruckten Bereich
    chapterprefix=true,     % Kapitel x vor dem Kapitelnamen
    cleardoublepage=plain]{scrbook}% {IEEEtran}

% Pakete einbinden, die benötigt werden
\usepackage{siunitx}	%added by user
\usepackage{tabularx}
\usepackage{multirow}
\usepackage{tikz}
\usepackage{adjustbox}
\usetikzlibrary{calc}
\usetikzlibrary{patterns}


%colors for the figures of the paper
% Some colors:
\tikzstyle{SType}=[circle,text=white,fill=blue,inner sep=2pt]
\tikzstyle{BType}=[circle,text=black,fill=orange!80,inner sep=2pt]
\tikzstyle{IType}=[circle,text=black,fill=black!20!green,inner sep=2pt]
\tikzstyle{UType}=[circle,text=black,fill=gray!50,inner sep=2pt]
\tikzstyle{edge from parent}=[draw,->]
\tikzstyle{level 1}=[sibling distance=8mm,level distance=10mm]

\newcommand{\celltype}[2]{\protect\tikz[baseline=(C.base)] \protect\node[#1] (C) {#2};}
\newcommand{\celltypeS}{\protect\tikz[baseline=(C.base)] \protect\node[SType] (C) {S};}
\newcommand{\celltypeB}{\protect\tikz[baseline=(C.base)] \protect\node[BType] (C) {B};}
\newcommand{\celltypeI}{\protect\tikz[baseline=(C.base)] \protect\node[IType] (C) {I};}
\newcommand{\celltypeU}{\protect\tikz[baseline=(C.base)] \protect\node[UType] (C) {U};}  
\newcommand{\etal}{\textit{et al.}}



\usepackage{scrpage2}
\usepackage[utf8]{inputenc}       % Dateien in UTF-8 benutzen
\usepackage[T1]{fontenc}          % Zeichenkodierung
\usepackage{graphicx}             % Bilder einbinden
\usepackage[main=ngerman, english]{babel}       % Deutsch und Englisch unterstützen
\usepackage{xcolor}               % Color support
\usepackage{amsmath}              % Matheamtische Formeln
\usepackage{amsfonts}             % Mathematische Zeichensätze
\usepackage{amssymb}              % Mathematische Symbole
\usepackage{float}                % Fließende Objekte (Tabellen, Grafiken etc.)
\usepackage{booktabs}             % Korrekter Tabellensatz
\usepackage[printonlyused]{acronym}  % Abkürzungsverzeichnis [nur verwendete Abkürzugen]
\usepackage{makeidx}              % Sachregister
\usepackage{listings}             % Source Code listings
\usepackage{listingsutf8}         % Listings in UTF8
\usepackage[hang,font={sf,footnotesize},labelfont={footnotesize,bf}]{caption} % Beschriftungen
\usepackage[scaled]{helvet}       % Schrift Helvetia laden
\usepackage[absolute]{textpos}	  % Absolute Textpositionen (für Deckblatt)
\usepackage{calc}                 % Berechnung von Positionen
\usepackage{blindtext}            % Blindtexte
\usepackage[bottom=40mm,left=35mm,right=35mm,top=30mm]{geometry} % Ränder ändern
\usepackage{setspace}             % Abstände korrigieren
\usepackage{ifthen}               % Logische Bedingungen mit ifthenelse
\usepackage{scrhack}              % Get rid of tocbasic warnings
\usepackage[pagebackref=false]{hyperref}  % Hyperlinks
\usepackage[all]{hypcap}          % Korrekte Verlinkung von Floats
\usepackage[autostyle=true,german=quotes]{csquotes}   % Zitate
\usepackage[backend=biber,
  isbn=false,                     % ISBN nicht anzeigen, gleiches geht mit nahezu allen anderen Feldern
  %sortlocale=de_DE,               % Sortierung der Einträge für Deutsch
  sortlocale=en_US,              % Sortierung der Einträge für Englisch
  autocite=inline,                % regelt Aussehen für \autocite (inline=\parancite)
  hyperref=true,                  % Hyperlinks für Ziate
  style=ieee                     % Zitate als Zahlen [1]
  %style=alphabetic               % Zitate als Kürzel und Jahr [Ein05]
  %style=authoryear                % Zitate Author und Jahr [Einstein (1905)]
]{biblatex}                       % Literaturverwaltung mit BibLaTeX
\usepackage{rotating}             % Seiten drehen

\setlength{\bibitemsep}{1em}     % Abstand zwischen den Literaturangaben
\setlength{\bibhang}{2em}        % Einzug nach jeweils erster Zeile

% Trennung von URLs im Literaturverzeichnis (große Werte [> 10000] verhindern die Trennung)
\defcounter{biburlnumpenalty}{10} % Strafe für Trennung in URL nach Zahl
\defcounter{biburlucpenalty}{500}  % Strafe für Trennung in URL nach Großbuchstaben
\defcounter{biburllcpenalty}{500}  % Strafe für Trennung in URL nach Kleinbuchstaben

% Farben definieren
\definecolor{linkblue}{RGB}{0, 0, 100}
\definecolor{linkblack}{RGB}{0, 0, 0}
\definecolor{comment}{RGB}{63, 127, 95}
\definecolor{darkgreen}{RGB}{14, 144, 102}
\definecolor{darkblue}{RGB}{0,0,168}
\definecolor{darkred}{RGB}{128,0,0}
\definecolor{javadoccomment}{RGB}{0,0,240}

% Einstellungen für das Hyperlink-Paket
\hypersetup{
    colorlinks=true,      % Farbige links verwenden       
%    allcolors=linkblue,
    linktoc=all,          % Links im Inhaltsverzeichnis
    linkcolor=linkblack,  % Querverweise
    citecolor=linkblack,  % Literaturangaben
	filecolor=linkblack,  % Dateilinks
	urlcolor=linkblack    % URLs
}

% Einstellungen für Quelltexte
\lstset{     
      xleftmargin=0.2cm,     
      basicstyle=\footnotesize\ttfamily,
      keywordstyle=\color{darkgreen},
      identifierstyle=\color{darkblue},
      commentstyle=\color{comment}, 
      stringstyle=\color{darkred}, 
      tabsize=2,
      lineskip={2pt},
      columns=flexible,
      inputencoding=utf8,
      captionpos=b,
      breakautoindent=true,
	  breakindent=2em,
	  breaklines=true,
	  prebreak=,
	  postbreak=,
      numbers=none,
      numberstyle=\tiny,
      showspaces=false,      % Keine Leerzeichensymbole
      showtabs=false,        % Keine Tabsymbole
      showstringspaces=false,% Leerzeichen in Strings
      morecomment=[s][\color{javadoccomment}]{/**}{*/},
      literate={Ö}{{\"O}}1 {Ä}{{\"A}}1 {Ü}{{\"U}}1 {ß}{{\ss}}2 {ü}{{\"u}}1 {ä}{{\"a}}1 {ö}{{\"o}}1
}

\urlstyle{same}

% Einstellungen für Überschriften
\renewcommand*{\chapterformat}{%
  \Large\chapapp~\thechapter   % Große Schrift
  \vspace{0.3cm}               % Abstand zum Titel des Kapitels
}

% Abstände für die Überschriften setzen
\renewcommand{\chapterheadstartvskip}{\vspace*{2.6cm}}
\renewcommand{\chapterheadendvskip}{\vspace*{1.5cm}}

\RedeclareSectionCommand[
  beforeskip=-1.8\baselineskip,
  afterskip=0.25\baselineskip]{section}

\RedeclareSectionCommand[
  beforeskip=-1.8\baselineskip,
  afterskip=0.15\baselineskip]{subsection}

\RedeclareSectionCommand[
  beforeskip=-1.8\baselineskip,
  afterskip=0.15\baselineskip]{subsubsection}


% In der Kopfzeile nur die kurze Kapitelbezeichnung (ohne Kapitel davor)
\renewcommand*\chaptermarkformat{\thechapter\autodot\enskip}
\automark[chapter]{chapter}

% Einstellungen für Schriftarten
\setkomafont{pagehead}{\normalfont\sffamily}
\setkomafont{pagenumber}{\normalfont\sffamily}
\setkomafont{paragraph}{\sffamily\bfseries\small}
\setkomafont{subsubsection}{\sffamily\itshape\bfseries\small}
\addtokomafont{footnote}{\footnotesize}
\setkomafont{chapter}{\LARGE\selectfont\bfseries}

% Wichtige Abstände
\setlength{\parskip}{0.2cm}  % 2mm Abstand zwischen zwei Absätzen
\setlength{\parindent}{0mm}  % Absätze nicht einziehen
\clubpenalty = 10000         % Keine "Schusterjungen"
\widowpenalty = 10000        % Keine "Hurenkinder"
\displaywidowpenalty = 10000 % Keine "Hurenkinder"
\renewcommand{\footnotesize}{\fontsize{9}{10}\selectfont} % Größe der Fußnoten
\setlength{\footnotesep}{8pt} % Abstand zwischen den Fußnoten

% Index erzeugen
\makeindex

% Einfacher Font-Wechsel über dieses Makro
\newcommand{\changefont}[3]{
\fontfamily{#1} \fontseries{#2} \fontshape{#3} \selectfont}

% Eigenes Makro für Bilder
\newcommand{\bild}[3]{
\begin{figure}[h]
  \centering
  \includegraphics[width=#2]{#1}
  \caption{#3}
  \label{#1}
\end{figure}}

% Wo liegt Sourcecode?
\newcommand{\srcloc}{src/}

% Wo sind die Bilder?
\graphicspath{{bilder/}}

% Makros für typographisch korrekte Abkürzungen
\newcommand{\zb}[0]{z.\,B.\ }
\newcommand{\dahe}[0]{d.\,h.\ }
\newcommand{\ua}[0]{u.\,a.\ }

% Flags für Veröffentlichung und Sperrvermerk
\newboolean{hsmapublizieren}
\newboolean{hsmasperrvermerk}
