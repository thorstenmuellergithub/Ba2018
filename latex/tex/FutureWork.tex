\chapter{Future work}
This chapter provides ideas for the future of the project. These ideas could have a positive impact on the research of bladder cancer.

\subsubsection{Evaluation of 2D simulations}
With the improvements of sections \ref{sec:calculationStpesUntilUrination}, \ref{sec:TargetVolumeSurfaceAfterMitosis} and \ref{sec:ApproximationError} and the findings of section \ref{sec:vD} the 2D simulations should be checked again. The changes of the named sections affect the simulation no matter if it is done in two or in three dimensions. Therefore, a check if the results of the 2D simulations are still the same might be appropriate. Additionally, it should be verified if these results are the same for a voxel density of one. If the results differ from the findings so far it might be important to redo the 2D simulations with a voxel density of one before continuing with the 3D technology.

\subsubsection{Hexagonal lattice type}
As the result in section \ref{sec:GrowSphereCells} reveals is not possible to keep a sphere shape. It might be a result of the square lattice and therefore the hexagonal lattice type should be tested if it suits this task better. With the hexagonal lattice the cells are hexagons in 2D and rhombic dodecahedrons in 3D. A rhombic dodecahedron is in its form already similar to a sphere. Hence, it might work to keep the sphere shape of a cell with these objects. \newline
In addition to the different lattice types it should be tested if another approximation to the surface and volume of a sphere is possible. A possible approximation is to use pyramids. With the use of the hexagonal lattice type the cells are made of rhombic dodecahedrons, in three dimensions. The rhombic dodecahedron consists of twelve polygons, where each of the twelve polygons is a pyramid. In case of a dodecahedron all parts have the same area as well as the same shape \cite{Horn1984}. An explanation of the rhombic dodecahedron and some basic calculations can be found at Wolfram MathWorld \cite{RhombicDodecahedron.html}.

\subsubsection{Malignant cells}
As an extension of the project malignant cells could be simulated. It might be a challenge to implement these cells, as these cells behave differently from the non-malignant cells. This could increase the reality of the simulation a lot. First for the hexagonal lattice should be tried to get sphere cells and a solid 3D simulation. It might be a useful extension if the 3D simulation works properly.

\subsubsection{\ac{CC3D}}
\ac{CC3D} showed some weaknesses during the bachelor thesis. It is used for several years in the project and therefore there is a good knowledge how to handle the weaknesses. Beside the weaknesses \ac{CC3D} has also its strengths. It allows to develop a simulation program in several programming languages.  This brings flexibility to the developers as they can choose if a more object oriented or a script oriented program is used. It would be very useful, especially for complex programs, if a debug during the simulation would be possible. Another strength of \ac{CC3D} is that it has a very active community. In the community a lot of questions are asked and members of the project as well as other community members help each other as much as possible. Therefore, it is possible to ask the community and get a feedback regarding the problem soon. \newline
\ac{CC3D} is not the only program which could be used. In a master thesis of a former student it was evidenced that it is the best suited program for the simulation of the urothelium \cite{MSCAngelo} at this time. This evaluation was made almost four years in the past. Therefore, it might be possible that nowadays there are other programs as well as other approaches which could be used. One possible program would be Morpheus. It is developed at the Technical University at Dresden. In the most recent paper, in which the program is used, a 3D multi-scale model of fluids in the liver was simulated \cite{Meyer2017}. Another program for the simulation might be Biocellion. It uses a parallel developing approach. This means the hardware sources of the computer can be used more as it is possible to let the cores of a CPU do several tasks \cite{Kang2014}. It might be suited for this problem because a cell has a sphere shape by default \cite{Kang2014}. These two programs could be an alternative to \ac{CC3D}. \newline
An approach which uses parallelization is the \ac{RW} algorithm \cite{Cercato2006}. This approach uses the \ac{CPM} algorithm. With the created \ac{RW} algorithm a speedup of 3 to 10 times in comparison the \ac{CPM} is possible \cite{Cercato2006}. It is also mentioned that it is not enough to have a fast algorithm which allows parallelization, the developer has also to know how to use the algorithm in order to develop a faster simulation. \newline
These programs and approaches might be an alternative to \ac{CC3D}. To determine if \ac{CC3D} is the best suited program for the morphogenesis simulation of the urothelium, it might be useful to do a new evaluation of the different algorithms and programs for cell simulation at the market.



%\begin{itemize}
%\item algorithmus weiterentwickeln
%\item (voxel Density)
%\item (adhesion)
%\end{itemize}