% Die Arbeit besteht aus Kapiteln (chapter)
\chapter{Done Stuff}

% Jedes Kapitel besteht aus Unterkapiteln (section)
\section{Week one 28.11-05.12}
\begin{itemize}
\item Einlesen
\item 3D Simulation auf eigenem laptop ausprobieren
\item probieren die Doku zu CC3D zu verstehen
\end{itemize}

\section{Week two 05.12-12.12}
\begin{itemize}
\item Coding Kommentieren
\item Code structurieren
\item Doku erstellen
\end{itemize}

\section{Week three 12.12-19.12}
\begin{itemize}
\item Grundlagen verstehen
\item Methoden mit angelo besprechen
\item Aufgabe (3D Simulation) überprüfen
\item Energiefunktionen verstehen
\item Literatur recherche 
\end{itemize}

\section{Week four 19.12-26.12}
\begin{itemize}
\item Literatur recherche
\item Python scripts auf ein lambda wert begrenzen
\item Changed the cell drawing (reduced parameters as they were redundant)
\item reduce approximation error  still a problem by 0.5 (at ExecConfig.calcPixelFromMuMeter and GrowthSteppable.moduroStep)
\item Checked GrowthSteppable
\item Tryied different cell drawing methods
\end{itemize}

\section{Week five 26.12-02.01}
\begin{itemize}
\item Structure the task - create goals for the BA
\item Created a table of contents
\item Studied GrowthSteppable (should be alright -> without changes to do for the 3D Simulation)
\item figured out: adhesion energy has some magic numbers -> by setting it to 0 cells stay in their more or less cubic form even when they touch
\item figured out: Volume-TargetVolume is wrong in one of the two cells after mitosis
\item figured out: Surface-TargetSurface is wrong in every cell after mitosis
\end{itemize}

\section{Week six 02.01-09.01}
\begin{itemize}
\item Added a while loop to the ModelConfig.setCellAttributes() in a way that the targetVolume and targetSurface is always at least 1 larger than the current Volume / Surface - during Mitosis there were wrong values between volume /surface and targetVolume / targetSurface
\item Changed the ExecConfig.calcVoxelVolumeFromVolume() method -> now we do not have rounding errors and calculate further with them - now we calculate the amount of voxels and then round up or down
\item Corrected the calculation of the amount of stem cell at the basal membrane
\end{itemize}

\section{Week seven 09.01-16.01}
\begin{itemize}
\item Thesis besprochen
\item An thesis geschrieben
\item 
\end{itemize}

\section{Week eight 16.01-23.01}
\begin{itemize}
\item ersten teil der thesis besprochen
\item an thesis geschrieben
\item 
\end{itemize}

\section{Week nine 23.01-30.01}
\begin{itemize}
\item 
\item 
\item 
\end{itemize}

\section{Week ten 30.01-06.02}
\begin{itemize}
\item 
\item 
\item 
\end{itemize}

\section{Week eleven 06.02-13.02}
\begin{itemize}
\item 
\item 
\item 
\end{itemize}

\section{Week twelf 13.02-20.02}
\begin{itemize}
\item 
\item 
\item 
\end{itemize}

\section{Week thirteen 20.02-27.02}
\begin{itemize}
\item 
\item 
\item 
\end{itemize}
% Unterkapitel können noch einmal durch subsections untergliedert 
% werden (jetzt auf der 3. Ebene)
\subsection{Dritte Ebene}

% Mit Labels können Sie sich später im Text wieder auf diese Stelle beziehen
\label{Gliederung:EbeneDrei}

% Einträge für den Index anlegen. Ein Index wird normalerweise in einer Abschluss
% Arbeit nicht benötigt.
\index{Gliederung!Ebenen}

% Auf der 4. Ebene liegen die subsubsections. In diesem Template bekommt die
% 4. Ebene keinen Nummern mehr und erscheint auch nicht im Inhaltsverzeichnis
\subsubsection{Vierte Ebene}

% Auf der 5. Ebene werden einzelne Absätze mit Überschriften versehen.
\paragraph{Fünfte Ebene} Anders als in diesem Beispiel, darf in Ihrer Arbeit kein Gliederungspunkt auf seiner Ebene alleine stehen. D.\,h. wenn es ein 1.1 gibt, muss es auch ein 1.2 geben.
