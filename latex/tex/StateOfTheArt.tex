\chapter{State of the Art}
%What was given to me
The current moduro project has a stable 2D simulation of 16 different models using \ac{CC3D}. With these models it is tried to make predictions about how bladder cancer arises. The simulations are performed by \ac{CC3D} and then several values, e.g. the fitness of the model or how realistic the model is, etc., are summarized and displayed by the 'Moduro-Toolbox'. The project consits of several models and several scripts, both are written in python, i.e. a programming language. The models include properties of the specific model, e.g. adhesion energy or the possibility of the new cell types after mitosis, i.e. cell division, and the scripts modify the cell behavior, e.g. they check when a mitosis takes place, how fast the cell will growth, etc.. The scripts are also calculating if the current model is a realistic one or not. These evaluating data are later used by the 'Moduro-Toolbox' to provide a overview over these data.

\section{Display and Simulation of the Urothelium}
%wie wird es dargestellt 
In \ac{CC3D} we use simulate an urothel of the size of \SI{200}{\micro\metre} for the x-axis and \SI{100}{\micro\metre} for the y-axis. The simulation of the urothel covers 2 years. At the first calculation step, \ac{MCS} 0, the simulation is initialized, i.e. the cells are drawn and placed. In order to provide a realistic simulation, i.e. the simulation is not to easy and not to hard, we have to calculate events which occur every \ac{MCS} and some events which are occur not every \ac{MCS}. The events, which are performed every \ac{MCS} are a) cell growth and b) the check for mitosis. Events which are not performed every \ac{MCS} are c) urination d) cell death e) transformation and f) mutation. These events will occur at some point in the simulation
\subsection{Events every \ac{MCS}}
\subsubsection{Cell Growth}
Every \ac{MCS} we calculate the growth of a cell. In the project currently the maximum possible growth of a cell is calculated.  We need this calculation for the relation volume and TargetVolume as well as for the relation volume, surface and TargetSurface. \ac{CC3D} calculates the volume itself, because the by \ac{CC3D} calculated volume or surface is a read-only value, i.e. it can not be overwritten. We set every \ac{MCS} a targetVolume and a $\lambda_{vol}$ at the initialization of the simulation. If we would not set up a targetVolume the effective energy would be 0 and the simulation would not start. Cell growth is also required to influence the shape of the cell, as we calculate the shape out of the volume and apply the calculated surface as TargetSurface and the lambda $\lambda_{sur}$ to the effective energy.

\subsubsection{Mitosis}
The verification if a cell divides or growth further is done every simulation step. Doing so allows us to define a very specific \ac{MCS} at which a cell divides, as there a 500 \ac{MCS} per day. The cells which are able to growth and as a result divide are a) stem cells b) basal cells and c)intermediate cells. The umbrella cells are more a product of mitosis than of cell division, if they would divide there would be two instead of one umbrella cell.

\subsection{Events not every \ac{MCS}}
\subsubsection{Urination}
Every 12 hours an urination is takes place. This is simulated in a way that for \textit{some specific cells} a necrosis flag is set, i.e. the simulation scripts let the cell shrink and then dissappear.

\subsubsection{Necrosis}
If a cell dies, i.e. necrosis takes place, a flag in the cell dictionary is set. Every 12 hours, 250 \ac{MCS}, we check if a cell dies or not. If so, the cell will shrink and finally dissappear.

\subsubsection{Transformation}
A transformation can take place if a basal cell divides into a basal and a intermediate cell. Because the intermediate cell has to be inside the strata, it immediately will be transformed to an umbrella cell, which are kind the barrier to the urin in the bladder.

\subsubsection{Mutation}
In the current project every day it is checked if a cell mutates or not. Every cell type has, dependant on the biological model, a propability if the cell mutates or not. If the mutates, currently the flag for necrosis is set and therfor the cell dissappears.



\section{Simulation Properties}
%scripts which influence the simulation, e.g. growthsteppable mitosisSteppable

\section{Models}
The 2D simulation of the urothel provided 16 different models. These model differ mostly in which cell types after the mitosis are created. The project divides the models into two domains, one has the id 'SSD', which stands for "stem cell-like division" \cite{Torelli2017} and means that every time a stem cell divides there will be one stem and one basal cell. The second domain has the id 'SPA', which stands for "stem cell population asymmetry" \cite{Torelli2017}. In the second domain the stem cell has a propability of 90\% that it will be one stem and one basal cells after mitosis. There is also the chance with a probability of 5\% that after mitosis there are two stem cells or two basal cells. This two domains and how the different models derive is displayed in figure XY.

%\begin{figure}
%	\center
%	\includegraphics[scale=0.35]{figures/16DifferentModels.png}
%	\caption{Display of the 16 different models in the project and how they are derived \cite{Torelli2017}}
%	\label{img:16DifferentModels}
%\end{figure}


\begin{table}
\begin{centering}
\par\end{centering}
\begin{centering}
\begin{tabular}{|c|c|lc|}
\hline 
Type & ID & Description & Model\tabularnewline
\hline 
\hline 
\multirow{2}{0.02\textwidth}{\begin{turn}{90}
Stem cells
\end{turn}} & SSD & Stem cell-like division & \begin{tikzpicture}[]
\node[SType] {S} [grow=right]
	child {node [SType]  {S}
	}
	child {node [BType]  {B}
	};
\end{tikzpicture}\tabularnewline
\cline{2-4} 
 & SPA & Stem cell population asymmetry & \begin{tikzpicture}[]
\node[SType] {S} [grow=right]
	child {node [SType]  {S}
	}
	child {node [SType]  {S}
	};
\node at (0.5,-1) {$p_s=0.05$};
\node[SType] at (2,0) {S} [grow=right]
	child {node [SType]  {S}
	}
	child {node [BType]  {B}
	};    
\node at (2.5,-1) {$p_a=0.90$};    
\node[SType] at (4,0) {S} [grow=right]
	child {node [BType]  {B}
	}
	child {node [BType]  {B}
	};    
\node at (4.5,-1) {$p_s=0.05$};        
\end{tikzpicture}
\tabularnewline
\hline 
\multirow{4}{0.02\textwidth}{\begin{turn}{90}
Basal cells\ \ \ \ \ \ \ \ \ \ \ \ \ \ \ \ \ \ \ \
\end{turn}} & BSD & Stem cell-like division in basal cell & 
\begin{tikzpicture}[]
\node[BType] {B} [grow=right]
	child {node [IType]  {I}
	}
	child {node [BType]  {B}
	};
\end{tikzpicture}
\tabularnewline
\cline{2-4} 
 & BPA & Basal cell population asymmetry & \begin{tikzpicture}[]
\node[BType] {B} [grow=right]
	child {node [BType]  {B}
	}
	child {node [BType]  {B}
	};
\node at (0.5,-1) {$p_s=0.05$};
\node[BType] at (2,0) {B} [grow=right]
	child {node [BType]  {B}
	}
	child {node [IType]  {I}
	};    
\node at (2.5,-1) {$p_a=0.90$};    
\node[BType] at (4,0) {B} [grow=right]
	child {node [IType]  {I}
	}
	child {node [IType]  {I}
	};    
\node at (4.5,-1) {$p_s=0.05$};        
\end{tikzpicture}\tabularnewline
\cline{2-4} 
 & BPCD & Proliferation and contact differentiation of basal cells & \begin{tikzpicture}[]
\node[BType] {B} [grow=right]
	child {node [BType]  {B}
	}
	child {node [BType]  {B}
	};
\end{tikzpicture} \hspace{1em} and \hspace{1em}
\begin{tikzpicture}[]
\node[BType] {B} [grow=right]
	child[dashed] {node [IType] {I} edge from parent node[above=0.4cm] {$\neg$ BM}
	};
\end{tikzpicture}\tabularnewline
\cline{2-4} 
 & BCD & Only contact differentiation of basal cells & 
\begin{tikzpicture}[]
\node[BType] {B} [grow=right]
	child[dashed] {node [IType] {I} edge from parent node[above=0.4cm] {$\neg$ BM}
	};
\end{tikzpicture}\tabularnewline
\hline 
\multirow{2}{0.02\textwidth}{\begin{turn}{90}
Intermediate cells
\end{turn}} & IPCD & Proliferation and contact differentiation of intermediate cells & \begin{tikzpicture}[]
\node[IType] {I} [grow=right]
	child {node [IType]  {I}
	}
	child {node [IType]  {I}
	};
\end{tikzpicture} \hspace{1em} and \hspace{1em}
\begin{tikzpicture}[]
\node[IType] {I} [grow=right]
	child[dashed] {node [UType] {U} edge from parent node[above=0.4cm] {M}
	};
\end{tikzpicture}
\tabularnewline
\cline{2-4} 
 & ICD & Only contact differentiation of intermediate cells & 
\begin{tikzpicture}[]
\node[IType] {I} [grow=right]
	child[dashed] {node [UType] {U} edge from parent node[above=0.4cm] {M}
	};
\end{tikzpicture}
\tabularnewline
\hline 
\end{tabular}
\par\end{centering}
\caption{\label{Cell-Lineage-Components}Possible proliferation and differentiation concepts for the three cell types stem (S), basal (B) and intermediate (I) cell (column Type). Column ID assigns a label for a specific proliferation or differentiation rule which refers to a specific model (column Model). Per cell type group one model can be chosen. In total we have $2\cdot 4 \cdot2 =16$ lineage models. Dividing cells are expressed by a plain line and transformation (differentiation) by a dashed line. Tranformation might happen either through contact (see ICD) or the loss of contact (see BCD).}
\end{table}

\section{Adhesion and cell sorting}
During morphogensis of a strata the cells not only growth, they also sort themself. In order for the cell sorting there have to be different adhesion values \cite{REF}, i.e. how strong the surface of two different cell types are holding together. In the project this is done with a matrix, which every of the 16 model has.  
The cell type 'Medium' is a \ac{CC3D} specific cell type and describe the space where no cells are in the simulation.

%\begin{figure}
%	\center
%	\includegraphics[scale=0.6]{figures/adhesionMatrix.png}
%	\caption{Adhesion values for the different connections between cell types. Small values represent large adhesion and large values represent small adhesion [check the CC3D manual if it is correct] \cite{Torelli2017}}
%	\label{img:adhesionMatrix}
%\end{figure}


\begin{table}
\begin{centering}
\begin{tabular}{|cc|c|c|c|c|c|c|}
\hline 
Cell type & & $V_{min}$ & $d_{min}$ & $V_{max}$ & $d_{max}$ & Volume & Surface\tabularnewline
\hline 
\hline 
Stem & \celltypeS & 268 & 8 & 523 & 10 & perfect & average\tabularnewline
\hline 
Basal & \celltypeB & 381 & 9 & 523 & 10 & important & average\tabularnewline
\hline 
Intermediate & \celltypeI & 905 & 12 & 1767 & 15 & important & poor\tabularnewline
\hline 
Umbrella & \celltypeU & 1767 & 15 & 3591 & 19 & important & poor\tabularnewline
\hline 
\end{tabular}
\par\end{centering}
\caption{\label{tab:Cell-properties}Cell properties. Volumes $V$ in $\mu$m$^{3}$, diameters $d$ in $\mu$m. Columns \sl Volume \normalfont and \sl Surface \normalfont are adjusted with the weights $\lambda_{V}$ and $\lambda_{A}$, respectively.}
\end{table}


\section{Cell Properties}
%parts of scripts about the cell
Each cell of cell type has several attributes, moreover each cell has an cell dictionary, in which additional attributes are stored. The properties regarding the cell type are likely what every cell in general has, e.g. min- max Diameter, min- max Volume, growth in \SI{}{\micro\metre} per day or time until apoptosis, i.e. cell death, etc.. Some of the properties are displayed in figure XY. The attributes in the cell dictionary are more likely for the simulation. Therefore, these are some attributes which we are using to make decisions, e.g. the current and expected live time, a flag for necrosis can be set here, etc..

%\begin{figure}
%	\center
%	\includegraphics[scale=0.6]{figures/volumeDiameterOfACell.png}
%	\caption{Min and max volumes and diameters for the different cell types as well as the importance of the volume and the surface of the specific cell type \cite{Torelli2017}}
%	\label{img:volumeDiameterOfACell}
%\end{figure}


\begin{table}
\begin{centering}
\begin{tabular}{|c|c||c|c|c|c|c|c|}
\hline 
\multicolumn{2}{|c||}{Types} & M & BM & \celltypeS & \celltypeB & \celltypeI & \celltypeU \tabularnewline
\hline 
\hline 
Medium & M & 0 & 14 & 14 & 14 & 14 & 4\tabularnewline
\hline 
Basal membrane & BM &  & -1 & 1 & 3 & 12 & 12\tabularnewline
\hline 
Stem cell & \celltypeS &  &  & 6 & 4 & 8 & 14\tabularnewline
\hline 
Basal cell & \celltypeB &  &  &  & 5 & 8 & 12\tabularnewline
\hline 
Intermediate cell & \celltypeI &  &  &  &  & 6 & 4\tabularnewline
\hline 
Umbrella cell & \celltypeU &  &  &  &  &  & 2\tabularnewline
\hline 
\end{tabular}
\par\end{centering}
\caption{\label{tab:Surface-tension-values}Surface tension values for the four cell types, the basal membrane (BM) and the medium (M). The values are according to CompuCell3D conventions: small values represent high adhesion, higher values less adhesion/greater repulsion.}
\end{table}

Each cell have a cell dictionary, where different values about the cell are saved. These values are a) exp-life-time, i.e. the expected live time of a cell until necrosis, i.e. cell death---in the simulation the cell shrinks and then dissappears---takes place b)necrosis c)DNA d)TurnOver e)colony f)id g)removed h)inhibited  i)min-max-volume j)normal-volume k)growth-factor l)life-time.


\section{Fitness functions}
In order to validate the simulated models, there are several scripts which check if the model is realistic or not. These scripts will check every day, every 500, \ac{MCS} if the model is realistic or not \cite{Torelli2017}. The result of these two fitness functions will be written into a file, and later read out by the moduro toolbox.

\subsection{Arrangement fitness function}
The arrangement fitness function ensure that the strata of the simulated urothelium has the correct order \cite{Torelli2017}, i.e. that the first layer on the basal membrane consits only of stem and basal cells \cite{REFS}, the next three to five layers consits only of intermediate cells \cite{REFS} and that there is one layer of umbrella cells \cite{REFS}.

\begin{equation} 
f_{a}^{*} = \begin{cases}
\dfrac{1}{(1-L_{B})+(lib-L_{I})+(1-L_{U})+1} & \text{if amount of layers > 0} \\
0 & \text{else 0}
\end{cases}
\end{equation}
In this equation $L_{B}$ and $L_{U}$ are boolean values, i.e. they have the value 0 or 1 \cite{Torelli2017}. They are 1 if the the first layer of cells consits only of basal or stem cells and if the most upper layer consits only of umbrella cells, otherwise they will be 0 \cite{Torelli2017}.
$lib$ is the amount of layers in betwenn the first and the last layers \cite{Torelli2017}. $L_{I}$ contains the amount of layers, which consits only intermediate cells \cite{Torelli2017}. Therfore, $lib-L_{I}$ describes the amount cells which are not in there intended layer \cite{Torelli2017}. \newline
The arrangement fitness function is calculated columnwise, every \SI{25}{\micro\metre}. After this calculation the average of all calculations of this function is calculated \cite{Torelli2017}.

\subsection{Volume fitness function}
This function calculates the relative volume regarding the current volume of the different cell types in the urothel. The relative amount of the different cell types should be: stem and basal cell = 10\%, intermediate cells = 67\% and umbrella cells = 23\% considering an average thickness of \SI{85}{\micro\metre} \cite{Torelli2017}. Therfor the formula is:
\begin{equation} 
f_{V_{i}} = \dfrac{1}{4 (\dfrac{V_{Si}-V_{Ii}}{V_{Si}})^2 + 1}
\end{equation}
$V_{Si}$ and $V_{Ii}$ describes the \textit{should} and the actual \textit{is} volume of a specific cell type $i$ \cite{Torelli2017}. 

\subsection{Overall fitness function}
The overall fitness function calculates the total fitness out of the volume and the arrangement fitness function. Therefor the average of both functions is calculated \cite{Torelli2017}. The function is the following:
\begin{equation} 
f(t_{i}) = \dfrac{f_{V}(t_{i})+f_{a}^{*}(t_{i})}{2}
\end{equation}
$t_{i}$ describes a specific time point, in \ac{MCS}, at which this calculation could be done. At the end of the simulation is calculated as the average of the overall fitness function \cite{Torelli2017}. Therefor, the formula is:
\begin{equation} 
f = \dfrac{1}{e+1} + \sum_{i=0}^{e}{f(t_{i})}
\end{equation}
$e+1$ indicating the amount of calculations of the overall fitness function \cite{Torelli2017}.



\subsection{Moduro Toolbox}
The purpose of the moduro toolbox is that we are able to evaluate a simulation. The toolbox itself was an bachelor thesis. All the data, which are written in to a file, e.g. cell birth and death by mitosis, data about the volume and arrangement fitness, etc., can be read out and analyzed with the moduro toolbox.
It is possible to create a short video out of the simulation screenshots. This video is able to display the complete simulation within a few minutes. Therefor a extra program is required.




\section{Simulation Scripts}
In order to not overload this chapter by presenting all scripts, I will present parts of the scripts which are important for the given task. A complete overview of the scripts is displayed at figure XY.

\subsection{Area of stem cells}
To provide optimal proliferation, i.e. to growth and multiply, of the cells, around 12\% of the basal membrane are used for the stem cells \cite{Torelli2017}. Of these 12\% the amount of stem cells is calculated and then they will be drawn. In the current project there is no specific calculation of the percentage area of stem cells on the basal membrane. Moreover, the calculation is done with a magic number, i.e. a number in the code without any explanation. So there is no mathematical evidence that the calculation is correct and it would additional work to change the calculation if the area of the stem cells on the basal membrane changes. This calculation works fine for two dimensions, since the result is around 12\%. For the third dimension there have to be adjustments made.
\begin{lstlisting}[language=Python, caption = cell Area]
        # Adds the stem cells throughout the basal membrane:
        cellDiameter = self.cellTypes[2].getAvgDiameter()
        stemCellFactor = 8 * cellDiameter
        if self.execConfig.dimensions == 2:
            noStemCells = int(self.execConfig.xLength / stemCellFactor)
        else:
            noStemCells = int(self.execConfig.xLength * self.execConfig.yLength /
                              (stemCellFactor * stemCellFactor))

\end{lstlisting}

\subsection{Position of the stem cells}
After the amount of stem cells on the basal membrane is calculated, the cells will be positioned randomly on it. To do so a random value for the x and z Position is calculated in a way, that the cell will not be at the edge of the lattice. If the stem cell would be close to the lattice, than the proliferation could not take place in an optimal way.
\begin{lstlisting}[language=Python, caption = stem cell position]
        for s in range(1, noStemCells + 1, 1):
            xPos = random.uniform(cellDiameter, self.execConfig.xLength - cellDiameter)
            zPos = random.uniform(cellDiameter, self.execConfig.zLength - cellDiameter)
            if self.execConfig.dimensions == 2:
                self._addCubicCell(2, xPos, 2, 0, cellDiameter, cellDiameter, 0, steppable)
            else:
                self._addCubicCell(2, xPos, 2, zPos, cellDiameter, cellDiameter, cellDiameter, steppable)
\end{lstlisting}                

\subsection{Cell Drawing}
The cells are drawn in a cubic way. For every of the three dimensions a start and endpoint is defined, see the following listing:
\begin{lstlisting}[language=Python, caption = cell Draw]
steppable.cellField[xPosDim:xPosDim + xLengthDim - 1,
					yPosDim:yPosDim + yLengthDim - 1,
					zPosDim:zPosDim + zLengthDim - 1] = cell
\end{lstlisting}
In this listing 'x,y,zPosDim' defines the start points and 'x,y,zPosDim + x,y,zLengthDim - 1' defines the end points of the cell. 

\subsection{MinMaxVolume}
In order that we are able to simulate mitosis a minimum and a maximum value regarding cell size is necesseray. This is done by the 'MinMaxVolume' in the cell Dictionary. These is done with a one dimensional array with two values, as it is displayed in the listing below:
\begin{lstlisting}[language=Python, caption = MinMaxVolume]
cellDict['min_max_volume'] = [self.execConfig.calcVoxelVolumeFromVolume(cellType.minVol),
								self.execConfig.calcVoxelVolumeFromVolume(cellType.maxVol)]
\end{lstlisting} 
Every cell type has its own minimum and maximum volume \cite{Torelli2017}. Since these values are saved in \SI{}{\micro\metre}, they have to be converted to the voxel unit. With these values, in voxel, it is possible to set boundaries for the specific cell, e.g. to calculate the volume when mitosis takes place or the maximal volume of a cell of a specific cell type.
\subsection{Volume and TargetVolume}
As explained earlier, in order that the simulation starts the effective energy is not allowed to be 0. As we are initializing every cell, we set the target volume of every cell to be 1 larger than the current volume. This has the effect, that the simulation starts. During the simulation we calculate the target volume out of the possible growth volume and the current volume.
\begin{lstlisting}[language=Python, caption = set target Volume and Surface of a cell]
cell.targetVolume = cell.volume + 1  # At the beginning, the target is the actual size.
# cell.targetVolume = cellDict['normal_volume'] # At the beginning, the target is the actual size.
cell.targetSurface = self.execConfig.calcVoxelSurfaceFromVoxelVolume(cell.targetVolume)
\end{lstlisting}
In the same context we calculated the target surface as well. This has the same reason as for the target volume. 

\subsection{Lambda TargetVolume and TargetSurface}
The lambda values for the volume and for the surface describe how much the deviation between the current and the should value is considered in the effective energy, see section 1.GGH.
In the project there are several places where these values are set. One place is just behind the target volume and target surface is set.
\begin{lstlisting}[language=Python, caption = set lambda volume and lambda surface]
cell.lambdaVolume = self.execConfig.calcVolLambdaFromVolFit(cellType.volFit)
cell.lambdaSurface = self.execConfig.calcSurLambdaFromSurFit(cellType.surFit)
\end{lstlisting}
At this place for each lambda value a function is called, these are displayed below, which has itself a multiplier for the multiplier of the specific volume or surface energy.
\begin{lstlisting}[language=Python, caption = functions to calculate the lambda multiplier for the effective energy of the volume and surface]
def calcSurLambdaFromSurFit(self, surFit):
    return 0.05 * surFit

def calcVolLambdaFromVolFit(self, volFit):
	return 1.0 * volFit
\end{lstlisting}

The values for the volume and surface fitness are also set in every cell. Since these values are never used as a cell property, I not go further into detail of these two values.





